 \chapter{\label{chap:intro}Conclusão}
 



Com o desenvolvimento deste trabalho, foi possível aquirir um maior conhecimento sobre os conceitos da Segurança da Informação e sobre os requisitos para elaboração de um SGSI que estão presentes na ISO 27001 e sobre os controles e objetivos de controle presentes na ISO 27002.  O TCC em questão é necessário, pois na literatura  \cite{caio2019}, foram encontradas poucas ou nenhuma referência sobre os controles da ISO no contexto de Ecossistema de Software Móvel. O tema ainda é pouco explorado, entretanto a pesquisa se demonstrou viável e muito importante.
 
O processo metodológico da pesquisa foi longo e desafiador, sendo necessário a leitura de artigos a respeito de metodologia de pesquisa e entrevistas com especialistas. Encontrar especialistas na área de desenvolvimento de dispositivos móveis foi uma tarefa complicada, devido a dificuldade de encontrar um profissional com anos de experiência com o mercado. A condução da entrevista foi particularmente desafiadora, pois o entrevistado muitas vezes desviava do foco das perguntas a serem feitas, dificultando o processo de identificação da relevâncias das questões levantadas. Um fator limitante para o engajamento de mais participantes na pesquisa pode ter se dado ao tamanho da \textit{survey}, que tinha um tempo médio de resposta de 15 minutos, além do assunto ser relativamente novo e a pesquisa ter sido realizado em uma situação de pandemia. 

Mesmo com muitos desafios ao longo do desenvolvimento do trabalhado, o resultado mostrou-se positivo, pois o processo metodológico foi estabelecido e executado, dando uma estrutura bem definida para a pesquisa, que obteve resultados interessantes, mesmo que com um possível viés, mostrando que muitos desenvolvedores se preocupam com a maioria dos controles selecionas, encontrados na ISO/IEC 27002.
 
 
Como estudo futuro, podem ser realizadas entrevistas com especialistas, a fim de se obter recomendações das questões que ficaram deficitárias, de uma resposta forte ou que não tenham sido consideradas como preocupação, para que seja possível gerar um guia de boas práticas para auxiliar os desenvolvedores de dispositivos móveis a entenderem melhor.
 




