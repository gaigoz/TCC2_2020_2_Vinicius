 \chapter{\label{chap:intro}Conclusão}
 

\todo[inline]{Conclusões do trabalho realizado, dificuldades, conhecimentos aprendidos, percepções}

Com o desenvolvimento deste trabalho, foi possível aquirir um maior conhecimento sobre os conceitos da Segurança da Informação e sobre os requisitos para elaboração de um SGCI que estão presentes na ISO 27001 e sobre os controles e objetivos de controle presentes na ISO 27002
 
 
 O processo metodológico da pesquisa foi desafiador, sendo necessário a leitura de artigos a respeito de entrevistas com especialistas, \textit{surveys} e sobre como elaborar orientações baseadas nos resultados da \textit{survey}. 
 
 A condução de uma entrevista foi particularmente desafiador, pois o entrevistado muitas vezes desviava do foco das perguntas a serem feitas, dificultando o processo de identificação da relevâncias das questões levantadas
 
 
 
 Um fator limitante para o engajamento de mais participantes na pesquisa pode ter se dado ao tamanho da \textit{survey}, que tinha um tempo médio de resposta de 15 minutos, além do assunto ser relativamente novo e a pesquisa ter sido realizado em uma situação de pandemia.
 
 
 
 O TCC em questão é necessário, pois na literatura  \cite{caio2019}, foram encontradas poucas ou nenhuma referência sobre os controles da ISO no contexto de Ecossistema de Software Móvel. O tema ainda é pouco explorado, entretanto a pesquisa se demonstra viável.
 
 
 \todo[inline]{Aqui posso falar de algumas coisas que vieram a limitar o tratabalho o que acham?}
 

 
 \todo[inline]{Propor uma outra direção ou um aprofundamento?}