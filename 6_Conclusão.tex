 \chapter{\label{chap:intro}Conclusão}
 



Com o desenvolvimento deste trabalho, foi possível aquirir um maior conhecimento sobre os conceitos da Segurança da Informação e sobre os requisitos para elaboração de um SGCI que estão presentes na ISO 27001 e sobre os controles e objetivos de controle presentes na ISO 27002.  O TCC em questão é necessário, pois na literatura  \cite{caio2019}, foram encontradas poucas ou nenhuma referência sobre os controles da ISO no contexto de Ecossistema de Software Móvel. O tema ainda é pouco explorado, entretanto a pesquisa se demonstrou viável.
 
O processo metodológico da pesquisa foi longo e desafiador, sendo necessário a leitura de artigos a respeito de metodologia de pesquisa e entrevistas com especialistas. Encontrar especialistas na área de desenvolvimento de dispositivos móveis foi uma tarefa complicada, devido dificuldade de encontrar um profissional com anos de experiência com o mercado. A condução da entrevista foi particularmente desafiador, pois o entrevistado muitas vezes desviava do foco das perguntas a serem feitas, dificultando o processo de identificação da relevâncias das questões levantadas. Um fator limitante para o engajamento de mais participantes na pesquisa pode ter se dado ao tamanho da \textit{survey}, que tinha um tempo médio de resposta de 15 minutos, além do assunto ser relativamente novo e a pesquisa ter sido realizado em uma situação de pandemia.
 
 
Como estudo futuro, podem ser realizadas entrevistas com especialistas, a fim de obter recomendações das questões que ficaram deficitárias de uma resposta forte ou não foram consideradas como preocupação, para que seja possível gerar um guia de boas práticas para auxiliar os desenvolvedores de dispositivos móveis.
 




