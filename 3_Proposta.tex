\chapter{\label{chap:intro}problemática e objetivo de pesquisa}


\section{Problema}

A segurança tem se tornado uma preocupação grande entre os usuários de dispositivos \textit{mobile} (ou móvel), porém essa preocupação varia de acordo com o sistema operacional utilizado pelo usuário. De acordo com o estudo de Reinfelder \cite{reinfelder2014differences}, usuários de Android são mais cientes a respeito de segurança e  privacidade, pois questões e permissões de privacidade são importantes para eles na decisão de instalar um novo aplicativo . A conclusão provisória mostrou que usuários de Android parecem ter mais consciência da privacidade do que os usuários do iOS, entretanto é necessária uma investigação mais aprofundada.

%imagem aqui

Alguns autores possuem resultados em comum a respeito das dificuldades da adoção de segurança de informação em organizações. Jaramilo \cite{jaramillo2013cross} comenta sobre a diferença de prioridades institucionais. Furnell \cite{furnell2009integrated} dividiu as categorias em fator humano, técnico e institucional. Disterer \cite{disterer2013} afirma que a segurança da informação deve ser organizacionalmente estabelecida na organização, para que medidas possam ser promovidas e estabelecidas para todos.


Castle \cite{castle2016let} avaliou os desafios de segurança em serviços financeiros digitais em
dispositivos móvel e verificou que os desenvolvedores entrevistados se preocupam com questões de segurança. Com isso o autor levantou a seguinte questão: \textit{“Quando desenvolvedores ou designers de produto querem aprender mais sobre opções de padrões para medidas de segurança, quais recursos eles devem usar? “}. O autor sugere o \textit{Stack Overflow} como fonte inicial de aprendizado, porém relata que a plataforma ainda oferece pouca consistência e padronização para boas práticas.

%ver como citar corretamente erdem do BIB
No estudo realizado por Steglich e colegas  \cite{caio2019} foi investigada a literatura no tópico de segurança da informação em organizações, com o foco na ISO 27000, mais especificamente na ISO 27001 e seus controles. Foi descoberto que apenas 34 dos 114 controles descritos na ISO 27001 estavam presentes na literatura de ecossistemas de software \textit{mobile}.

Assim, no contexto de ecossistemas de software móvel, este TCC  buscou identificar se desenvolvedores externos ou internos adotam controles da ISO 27002 para o desenvolvimento de software \textit{mobile}.
\todo[inline]{tirei a parte do guia aqui, esta ok?}



\section{Objetivo geral}

O principal objetivo deste TCC é identificar por meio de uma \textit{survey} quais são os controles da ISO 27002 adotados no desenvolvimento de software \textit{mobile}. Com o resultado da \textit{survey} foi feita uma análise dos resultados para entender qual eram as questões que os desenvolvedores levavam em consideração e quais eles não levavam no momento de criar aplicações móveis.

\todo[inline]{modifiquei pra atender o novo objetivo, esta ok?}
 
\subsubsection{\textbf{Objetivos específicos:}}

Para atingir o objetivo geral deste TCC, definiu-se os seguintes objetivos específicos:

\begin{enumerate}
    \item Aprofundar o conhecimento sobre a literatura em segurança de informação baseado na ISO 27002 em ecossistemas de software \textit{mobile}.
    
    \item Definir e conduzir uma \textit{survey} baseado na revisão da literatura anterior.
    
    \item Analisar os resultados da \textit{survey} com o objetivo de compreender a preocupação dos desenvolvedores com as questões.
    
    
\end{enumerate}
%Quais são? teria que abrir o objetivo geral?


      

%\section{Cronograma}

%O cronograma foi dividido entre os 2 semestres do ano de 2020, contemplando a metodologia \textit{survey}. No primeiro semestre será elaborada uma análise critica da ISO 27000 separando as seções que sejam relevantes para o objetivo trabalho. No segundo semestre as seções serão validadas com especialistas para elaborar a \textit{survey} que irá gerar resultados  para elaborar um site com recomendações de boas práticas.

%\begin{figure}
%    \centering
%   \includegraphics[scale=0.59]{fig/crongrama.JPG}
%  \caption{Cronograma}
%   \label{fig:my_label}
%\end{figure}





%----------------------------------------------------------------------------------
% Exemplo do ambiente para citações diretas com mais de 3 linhas.
%
% Para citações diretas de até 3 linhas, faça assim:
% De acordo com Direto~(2011, p.~21): "bla bla bla, bla 'bla'."