%----------------------------------------------------------------------------------
% Exemplo do uso da classe tcc.cls. Veja o arquivo .cls
% para mais detalhes e instruções.
%----------------------------------------------------------------------------------
\chapter{\label{chap:intro}Introdução}
% Comando para inserir siglas. Tanto as siglas quanto as
% abreviaturas devem aparecer em ordem ALFABÉTICA nas listas
% correspondentes. Como a classe no momento não é capaz de ordenar
% as entradas automaticamente, existem duas alternativas:
%
%    a- Insira todas as siglas e abreviaturas no começo do texto,
%    manualmente e em ordem alfabética.
%
%    b- Caso esteja em um ambiente UNIX (Linux, Mac ou Cygwin/similares),
%    utilize o script sort.sh e o makefile que acompanham a
%    classe. O makefile automaticamente compila a monografia para
%    PDF (mas assume que o latex está acessível pela linha de
%    comando). Neste caso a ordenação é feita de forma automática.
\sigla{DDoS}{Ataque de negação de serviço}
\sigla{ECOSM}{Ecossistema de software móvel }
\sigla{IEC}{International Electrotechnical Commission}
\sigla{ISO}{Organização Internacional de Normalização}
\sigla{ICP-Brasil}{Infraestrutura de Chaves Públicas Brasileira}
\sigla{LGPD}{Lei Geral de Proteção de Dados Pessoais}
\sigla{SGSI}{Sistema de Gestão da Segurança da Informação}



%
% Comando para inserir abreviaturas.
%
\abrev{Abrev}{Abreviatura}
\abrev{Inform}{Informática}
%
% Comando para inserir símbolos. Estes irão aparecer em ordem
% de ocorrência, já que o número da página está presente na lista
% de símbolos.
\simbolo{Hz}{Hertz}
\simbolo{$\pi$}{Constante com valor aproximado de $3.1415926$}%
%
   

A informação é um ativo muito importante para organizações e necessita da proteção adequada. A segurança da informação é obtida por meio da adoção de normas e práticas reconhecidas no mercado como boas práticas. A Organização Internacional de Normalização (ISO) é uma organização não governamental responsável por definir normas e padrões internacionais. Para padrões que se enquadram no escopo da segurança da informação, a ISO é a responsável pela criação da série (ISO) 27000 \cite{caio2019}. Sobre o título de “Tecnologia da informação - Técnicas de segurança”, essa série é responsável por descrever requisitos de Sistema de Gestão de Segurança da Informação (SGSI), além de emitir certificações \cite{disterer2013}. Com a certificação ISO de um SGSI uma organização é capaz de transmitir a sociedade uma imagem positiva sobre seu gerenciamento da segurança. Disterer \cite{disterer2013} afirma que a segurança da informação deve estar organizacionalmente ancorada na empresa para que medidas de segurança possam ser promovidas e estabelecidas. 

% E as outras ISOs da familia 27000 são encontradas na literatura de MSECO?
%[]verificaram que apenas 34 dos 114 controles da ISO 27001 estavam presentes na literatura de Ecossistemas de Software móvel.

Neste Trabalho de Conclusão de Curso (TCC) identificou-se quais controles da ISO 27002 são adotados no desenvolvimento de software \textit{mobile} e quais não costumam ser levados em consideração.

%e quais são as dificuldades são as dificuldades ao tentar adotá-los.
%Também foram feitas recomendações de como sobrepor essas dificuldades. Com as recomendações baseadas nos resultados de uma \textit{Survey}, foi elaborado um guia em formato de \textit{site} de boas práticas na adoção de controles de segurança da informação no desenvolvimento de software \textit{mobile}.

O restante deste volume deste TCC está organizado da seguinte forma: O Capítulo 2 apresenta a fundamentação teórica na área da ISO 27000, 27001 e 27002 relatando seu uso e importância, também é discorrido o assunto sobre segurança em ecossistemas de software móveis. O Capítulo 3 detalha a proposta da pesquisa, contento o objetivo geral juntamente com os específicos e o cronograma. O capitulo 4 detalha o processo e a metodologia de pequisa utilizada neste trabalho que está dividia em 2 etapas, revisão da literatura, entrevista com especialistas e a \textit{survey}. O Capítulo 5 que mostra a análise dos resultados obtidos na \textit{surrvey}. E por fim no capítulo 6 é feita a discussão sobre os resultados.








 
